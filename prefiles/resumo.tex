%-------------------------------------------------------------------------------------------
% Tesetex é um software livre: você pode redistribuí-lo e/ou modificá-lo
% nos termos da Licença Pública Geral GNU 3, publicada pela Free Software Foundation.
% Tesetex é distribuído na esperança de que seja útil para você, mas SEM QUALQUER GARANTIA. 
% Veja a Licença Pública Geral GNU para mais detalhes.
% Você provavelmente recebeu uma cópia da Licença Pública Geral GNU junto com o Tesetex. 
% Caso contrário, consulte <http://www.gnu.org/licenses/>.
% Criado por Gustavo Silva Rodrigues e a comunidade LaTeX.
% Instruções completas disponíveis em: https://github.com/gusirosx/Tesetex
%-------------------------------------------------------------------------------------------
%==============================================================================================
%                                          Resumo
%==============================================================================================
\clearpage
\begin{center}
	\chapter*{Resumo}
\end{center}
\vspace{24pt}
\onehalfspacing
\noindent
SOBRENOME, Nome. Título da tese título da tese título da tese título da tese título da tese título da tese. \pageref{LastPage}p. Tese de Doutorado. Faculdade de Engenharia Mecânica, Universidade Federal de Uberlândia, Uberlândia, 2020.\\

Elemento obrigatório elaborado conforme NBR6028 da Associação Brasileira de Normas Técnicas (2003), apresentado em um só bloco de texto sem recuo de parágrafo consistindo na apresentação concisa das ideias do texto completo. Deve descrever de forma clara e sintética a natureza do trabalho, o objetivo, o método, os resultados e as conclusões, visando fornecer elementos para o leitor decidir sobre a consulta ao texto completo. Deve ser redigido em linguagem clara e objetiva, ser inteligível por si mesmo, empregar verbos na voz ativa e na terceira pessoa do singular e conter de \textbf{150 a 500 palavras}. Deve-se evitar o uso de símbolos abreviaturas, fórmulas, quadros, equações. Após o texto do resumo, seguem as palavras-chave representativas do conteúdo do trabalho, que devem aparecer após um espaço em branco de 1,5, à margem esquerda, separadas entre si por ponto final.
\\

\noindent
\emph{\textbf{Palavras-chave}}: Palavra1, Palavra2, Palavra3, Palavra4, Palavras5.\\
%==============================================================================================
%                                        Abstract
%==============================================================================================
\clearpage
\begin{center}
	\chapter*{Abstract}
\end{center}
\vspace{24pt}
\noindent
LASTNAME, Name. Thesis title thesis title thesis title thesis title thesis title thesis title thesis title. \pageref{LastPage}p. Ph.D. Thesis. School of Mechanical Engineering, Federal University of Uberlândia, Uberlândia, 2020.\\

Write here the English version of your "Resumo". Example text, example text, example text, example text, example text, example text, example text, example text, example text, example text, example text, example text, example text, example text, example text, example text, example text, example text, example text, example text, example text, example text, example text, example text, example text, example text, example text, example text, example text, example text, example text, example text, example text, example text, example text, example text, example text, example text, example text, example text, example text, example text, example text, example text, example text, example text, example text. Example text, example text, example text, example text, example text, example text, example text, example text, example text, example text, example text, example text, example text, example text, example text, example text, example text, example text, example text, example text, example text, example text.
\\

\noindent
\emph{\textbf{Keywords}}: Keyword1, Keyword2, Keyword3, Keyword4, Keyword5.