%===================================================================================================
%                                       Capítulo 5 
%===================================================================================================
\chapter{Resultados e Discussões}\label{cap5}

Nessa etapa são comparados, avaliados e criticados os resultados. Discute-se o valor absoluto e relativo dos resultados. Da apresentação dos fatos pode-se passar para deduções paralelas, generalização cautelosa e enumeração das questões que ocorrem do autor para as quais não encontrou resposta e que requerem estudos e pesquisas além do limite do trabalho. 

Pode ser elaborada uma única seção para “Resultados e discussões” em caso de textos não muito extensos e a discussão não muito detalhada. Em caso de resultados complexos é pertinente elaborar um texto para apresentar os resultados e  outro para discussão a partir do que foi constatado.  

A apresentação dos resultados deve ser clara, objetiva, lógica e acompanhada de comentários. A apresentação de tabelas e ilustrações facilita de maneira extraordinária esta parte do texto. Os dados obtidos, mesmo quando em grande quantidade, devem fazer parte do trabalho, mesmo que em forma de apêndice.
