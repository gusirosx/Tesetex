%-------------------------------------------------------------------------------------------
% Tesetex é um software livre: você pode redistribuí-lo e/ou modificá-lo
% nos termos da Licença Pública Geral GNU 3, publicada pela Free Software Foundation.
% Tesetex é distribuído na esperança de que seja útil para você, mas SEM QUALQUER GARANTIA. 
% Veja a Licença Pública Geral GNU para mais detalhes.
% Você provavelmente recebeu uma cópia da Licença Pública Geral GNU junto com o Tesetex. 
% Caso contrário, consulte <http://www.gnu.org/licenses/>.
% Criado por Gustavo Silva Rodrigues e a comunidade LaTeX.
% Instruções completas disponíveis em: https://github.com/gusirosx/Tesetex
%-------------------------------------------------------------------------------------------
%===================================================================================================
%                                    Lista de Abreviaturas
%===================================================================================================
\clearpage
\phantomsection
\addcontentsline{toc}{chapter}{Lista de Abreviaturas e Siglas}
\begin{center}
	\chapter*{Lista de Abreviaturas e Siglas}
\end{center}


\DTLnewdb{acronyms}
\abrev{UNIFAES}{Esquema de abordagem finita unificada do tipo exponencial, do inglês \textit{Unified Finite Approaches Exponential-type Scheme}}
\abrev{CDS}{Esquema da diferença central, do inglês \textit{Central Differencing scheme}}
\abrev{ADI}{Direção implícita alternada, do inglês \textit{Alternating Direction Implicit}}
\abrev{CFD}{Dinâmica dos fluidos computacional, do inglês \textit{Computational Fluid Dynamics}}
\abrev{CFL}{\textit{Courant Friedrichs Lewy} }
\abrev{DNS}{Simulações numéricas diretas, do inglês \textit{direct numerical simulations}}
\abrev{FOU}{\textit{Upwind} de primeira ordem, do inglês \textit{First Order Upwind} }
\abrev{LOADS}{Esquema de diferenciação localmente analítico, do inglês \textit{Locally Analytic Differencing Scheme} }
\abrev{MAC}{\textit{Marker and Cell}}
\abrev{MDF}{Métodos das diferenças finitas}
\abrev{MEF}{Métodos dos Elementos Finitos}
\abrev{MVF}{Métodos dos Volumes Finitos}
\abrev{QUICK}{Interpolação quadrática a montante para a cinemática convectiva, do inglês \textit{Quadratic Upstream Interpolation for Convective Kinematics} }
\abrev{QUICKER}{Interpolação quadrática a montante para a cinemática convectiva estendida e revisada, do inglês \textit{Quadratic Upstream Interpolation for Convective Kinematics Extended and Revised}}
\abrev{RMS}{\textit{Root Mean Square} }
\abrev{IBGE}{Instituto Brasileira de Geografia e Estatistica}
\abrev{SIMPLEC}{\textit{Semi-Implicit Method for Pressure-Linked Equation Consistent} }
\abrev{CNPq}{Conselho Nacional de Desenvolvimento Científico e Tecnológico }
\abrev{SOU}{\textit{Upwind} de segunda ordem, do inglês \textit{Second Order Upwind} }
\abrev{ABNT}{Associação Brasileira de Normas Técnicas}
\abrev{TDMA}{\textit{Tridiagonal Matrix Algorithm} }
% Sort the database
\DTLsort*{Acronym}{acronyms}

% Display the contents of the database acronyms
\begin{abbreviations}
	\DTLforeach*{acronyms}{\thisAcronym=Acronym,\thisDesc=Description}{\item[\thisAcronym] \thisDesc}
\end{abbreviations}




