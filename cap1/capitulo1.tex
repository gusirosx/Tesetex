%-------------------------------------------------------------------------------------------
% Tesetex é um software livre: você pode redistribuí-lo e/ou modificá-lo
% nos termos da Licença Pública Geral GNU 3, publicada pela Free Software Foundation.
% Tesetex é distribuído na esperança de que seja útil para você, mas SEM QUALQUER GARANTIA. 
% Veja aLicença Pública Geral GNU para mais detalhes.
% Você provavelmente recebeu uma cópia da Licença Pública Geral GNU junto com o Tesetex. 
% Caso contrário, consulte <http://www.gnu.org/licenses/>.
% Criado por Gustavo Silva Rodrigues e a comunidade LaTeX.
% Instruções completas disponíveis em: https://github.com/gusirosx/Tesetex
%-------------------------------------------------------------------------------------------
%===================================================================================================
%                                       Capítulo 1 
%===================================================================================================
\chapter{Introdução}\label{cap1}

Este \textit{template} apresenta as regras básicas para a elaboração do trabalho segundo as normas ABNT. Além das regras básicas previstas aqui, solicita-se consultar outros detalhes da norma ABNT sempre que se desejar inserir ou configurar algum elemento não previsto aqui. Ou seja, mesmo que este \textit{template} não preveja as demais regras ABNT, por ser uma visão simplificada, ainda assim elas precisam ser seguidas. 

%---------------------------------------------------------------------------------------------------
\section{Objetivos}\label{cap11}

A descrição dos objetivos deve ser precisa e clara permitindo ao leitor a compreensão do objetivo geral e dos objetivos específicos deste trabalho. A diagramação fica a critério do aluno, considerando a clareza de apresentação. Os objetivos podem ser apresentados em listas ou texto corrido.

Texto de exemplo, texto de exemplo, texto de exemplo, texto de exemplo, texto de exemplo, texto de exemplo, texto de exemplo, texto de exemplo, texto de exemplo, texto de exemplo, texto de exemplo, texto de exemplo, texto de exemplo, texto de exemplo, texto de exemplo, texto de exemplo, texto de exemplo, texto de exemplo, texto de exemplo, texto de exemplo, texto de exemplo, texto de exemplo, texto de exemplo.Texto de exemplo, texto de exemplo, texto de exemplo, texto de exemplo, texto de exemplo, texto de exemplo, texto de exemplo, texto de exemplo, texto de exemplo, texto de exemplo, texto de exemplo, texto de exemplo, texto de exemplo, texto de exemplo, texto de exemplo, texto de exemplo, texto de exemplo, texto de exemplo, texto de exemplo, texto de exemplo, texto de exemplo, texto de exemplo, texto de exemplo. Texto de exemplo, texto de exemplo, texto de exemplo, texto de exemplo, texto de exemplo, texto de exemplo, texto de exemplo, texto de exemplo, texto de exemplo, texto de exemplo, texto de exemplo, texto de exemplo, texto de exemplo, texto de exemplo, texto de exemplo, texto de exemplo, texto de exemplo, texto de exemplo, texto de exemplo, texto de exemplo, texto de exemplo, texto de exemplo, texto de exemplo.

\section{Justificativa}\label{cap12}

A descrição das justificativas deve ser precisa e clara permitindo ao leitor a compreensão da relevância da realização deste trabalho.

