%-------------------------------------------------------------------------------------------
% Tesetex é um software livre: você pode redistribuí-lo e/ou modificá-lo
% nos termos da Licença Pública Geral GNU 3, publicada pela Free Software Foundation.
% Tesetex é distribuído na esperança de que seja útil para você, mas SEM QUALQUER GARANTIA. 
% Veja a Licença Pública Geral GNU para mais detalhes.
% Você provavelmente recebeu uma cópia da Licença Pública Geral GNU junto com o Tesetex. 
% Caso contrário, consulte <http://www.gnu.org/licenses/>.
% Criado por Gustavo Silva Rodrigues e a comunidade LaTeX.
% Instruções completas disponíveis em: https://github.com/gusirosx/Tesetex
%-------------------------------------------------------------------------------------------
%===================================================================================================
%                                    Lista de Símbolos
%===================================================================================================
\clearpage
\phantomsection
\addcontentsline{toc}{chapter}{Lista de Símbolos}
\renewcommand*\nomname{}{\begin{center} \chapter*{Lista de Símbolos} \end{center}}
\vspace{-13ex}
%===================================================================================================

%[A] : Letras Latinas
%[B] : Letras Gregas
%[C] : Constantes Físicas
%[D] : Sobrescritos
%[E] : Subscritos
%[F] : Operadores Matemáticos

\nomenclature[E]{$w$}{Paredes}

\nomenclature[A]{$a_i$}{Coeficientes de influencia das equações do momento discretizadas}
\nomenclature[A]{$A_i$}{Coeficientes de influência modificados}
\nomenclature[A]{$C_f$}{Coeficiente de atrito}
\nomenclature[A]{$D$}{Dilatação, divergente de velocidade}
\nomenclature[A]{$ER$}{Razão de expansão do canal}
\nomenclature[A]{$D_i$}{Termo da difusão adimensional}
\nomenclature[A]{$F_i$}{Fluxo de massa convectiva}
\nomenclature[A]{$h$}{Altura do degrau}
\nomenclature[A]{$h_E$}{Altura do comprimento de entrada do degrau \nomunit{$[m]$}}
\nomenclature[A]{$h$}{Altura do degrau}
\nomenclature[A]{$h$}{Altura do degrau}
\nomenclature[A]{$h$}{Altura do degrau}
\nomenclature[A]{$h$}{Altura do degrau}
\nomenclature[A]{$h$}{Altura do degrau}
\nomenclature[A]{$h$}{Altura do degrau}
\nomenclature[A]{$h$}{Altura do degrau}
\nomenclature[A]{$h$}{Altura do degrau}
\nomenclature[A]{$h$}{Altura do degrau}
\nomenclature[A]{$h$}{Altura do degrau}
\nomenclature[A]{$H$}{Altura total do canal}
\nomenclature[A]{$L_c$}{Comprimento característico}
\nomenclature[A]{$P$}{Pressão física, mais carga hidrostática}
\nomenclature[A]{$J$}{Fluxos viscosos e advectivos combinados através do contorno da célula}
\nomenclature[A]{$K$}{Termo fonte da equação geradora}
\nomenclature[A]{$Re$}{Número de Reynolds}
\nomenclature[A]{$Pe$}{Número de Peclet}
\nomenclature[A]{$S$}{Termo fonte da equação de transporte}
\nomenclature[A]{$S_a$}{Termo fonte da equação de transporte discretizada}
\nomenclature[A]{$t$}{Tempo adimensional}
\nomenclature[A]{$u$}{Componente de velocidade na direção x}
\nomenclature[A]{$v$}{Componente de velocidade na direção y}
\nomenclature[A]{$V_c$}{Velocidade característica}
\nomenclature[A]{$X_i/h$}{Comprimentos característicos de cada região de recirculação}
\nomenclature[B]{$\phi$}{Variável genérica}
\nomenclature[B]{$\Gamma$}{Difusividade}
\nomenclature[B]{$\pi$}{Função real para o número de Peclet celular}
\nomenclature[B]{$\Pi$}{Coeficientes de influência no esquema de Allen e Southwell}
\nomenclature[B]{$\lambda$}{Autovalor da Solução elementar}
\nomenclature[B]{$\chi$}{Função real para o número de Peclet celular}
\nomenclature[B]{$\psi$}{Termo de correção para as equações do momentum discretizadas}
\nomenclature[B]{$\theta$}{ângulo entre a malha e o escoamento}
\nomenclature[B]{$\rho$}{Densidade do fluido}
\nomenclature[B]{$\tau_w$}{Tensão de cisalhamento na parede}
\nomenclature[B]{$\mu$}{Viscosidade dinâmica do fluido}
\nomenclature[B]{$\nu$}{Viscosidade cinemática do fluido}
\nomenclature[B]{$\omega$}{Sub-relaxação adotada no método de Gauss-Siedel}
\nomenclature[C]{$c$}{Velocidade da luz em um sistema inercial \nomunit{$299,792,458\, m/s$}}
\nomenclature[C]{$h$}{Constante de Plank \nomunit{$6.62607 \times 10^{-34}\, Js$}}
\nomenclature[C]{$g$}{Constante Gravitacional\nomunit{$6.67384 \times 10^{-11}\, N \cdot m^2/kg^2$}}

\printnomenclature
